\RequirePackage[l2tabu, orthodox]{nag}
\documentclass[11pt,uplatex,a4paper]{jsarticle}
%% file: fem2dp-2.tex
%% author: Junichi Motohisa
%% created: 2017-11-21 18:32:34
%% last saved : Time-stamp: <Sat Sep 14 20:06:54 JST 2019>
\usepackage[dvipdfmx]{graphicx}
\usepackage{multicol}
\usepackage{cite}
%\usepackage{mathptmx}
%\usepackage{color}
\usepackage{lscape}
% \usepackage{dcolumn}
\usepackage{moreverb}
% \usepackage[a4paper,margin=1in]{geometry}
\usepackage[driver=dvipdfmx,truedimen]{geometry}
% \geometry{left=30truemm,right=25truemm,top=25truemm,bottom=25truemm}
\geometry{margin=1truein}
% \usepackage[a4paper,margin=1in]{geometry}

%% taken from https://github.com/ryseto/emacs_and_tex/blob/master/template.tex
%% begin
\usepackage{amsmath,amsfonts,amsthm}
\usepackage{bm}
\usepackage[protrusion=true,expansion=true]{microtype}
\usepackage{siunitx}
\usepackage{xcolor}
%\usepackage[pdftex]{graphicx}
%\usepackage[pdftex,bookmarks,colorlinks]{hyperref}
% \usepackage[dvipdfmx]{hyperref}
% \usepackage{pxjahyper}
\usepackage[english]{babel}
\usepackage[T1]{fontenc}
\usepackage[utf8]{inputenc}
\usepackage{newtxtext}
\usepackage[varg]{newtxmath}
\usepackage{booktabs}
\usepackage{datetime}
\usepackage{setspace}
\usepackage{enumitem}
\usepackage{cleveref}
\usepackage[numbers]{natbib}

% \bibliographystyle{abbrvnat}
\setlist{itemsep=-3pt}
%% end

\def\d#1#2{\frac{d #1}{d #2}}
\def\dd#1#2{\frac{d^2 #1}{d #2^2}}
\def\pd#1#2{\frac{\partial #1}{\partial #2}}
\def\pdd#1#2{\frac{\partial^2 #1}{\partial #2^2}}
\def\Ene{\varepsilon}
\def\DOS{\mathcal{D}}
\def\vec#1{{\bm{#1}}}
\def\kvec{\bm{k}}
\def\kvecp{{\bm{k}'}}
\def\subvec#1{{\boldsymbol{\small #1}}}
 
\def\GP{{\vec{G}_\parallel}}
\def\GPP{{\vec{G}^\prime_\parallel}}
\def\KP{{\vec{k}_\parallel}}
\def\KP{{\vec{k}_\parallel}}
\def\XP{{\vec{x}_\parallel}}

\def\EneK{\varepsilon_{\kvec}}
\def\EneKp{\varepsilon_{\kvec^{'}}}
\def\EneKn{\varepsilon_{\kvec n}}
\def\EneKnp{\varepsilon_{\kvec^{'} n^{'}}}
\def\EneKm{\varepsilon_{m \vec{k}}}
\def\EneKmp{\varepsilon_{m' \vec{k}'}}

\def\bra#1{{\langle #1 |}}
\def\ket#1{{| #1 \rangle}}
\def\braket#1#2{{\langle #1 | #2 \rangle}}
\def\av#1{{\langle #1 \rangle}}

\def\sinc{\mathrm{sinc}}
\def\sgn{\mathrm{sgn}_{n,p}}
\def\Tr{{\mathrm{Tr}\, }}

\def\micon{\mathrm{\mu m}}
\def\nm{\mathrm{nm}}

\begin{document}

\title{Title}
\author{J.~Motohisa\\Graduate School of IST, Hokkaido University}
\date{\shortdate\today \, \ampmtime }
\maketitle

\section{Introduction}

\section{Ballistic FET in 2D}
Top of the Barrierのエネルギー$E_0$は
\begin{equation}
 E_0 + \alpha_D q V_{DS} + \alpha_G q V_{GS} -
  q^2\frac{n_{2D}(E_F - E_0)+n_{2D}(E_F-E_0-qV_{DS})}{2 C_{eff}}=0
  \label{eqn:tob}
\end{equation}
を解くことによって求めることができる。
ここで
\begin{equation}
 n_{2D}(E) = \mathcal{D}_{2D}k_B T \ln (1+\exp(\frac{E}{k_B T}))
\end{equation}
であり、また$\mathcal{D}_{2D}$は2次元の状態密度であり$\mathcal{D}_{2D}=m^*/\pi \hbar^2$である。

  \subsection{近似}
式(\ref{eqn:tob})を近似すると
\begin{equation}
 E_0 + \alpha_D q V_{DS} + \alpha_G q V_{GS} -
  q^2\frac{\mathcal{D}_{2D}(E_F - E_0)+\mathcal{D}_{2D}(E_F-E_0-qV_{DS})}{2 C_{eff}}=0
\end{equation}
と書ける。
 この式を解くと
 \begin{equation}
  \frac{E_0}{q} = -\frac{\alpha_G V_{GS} + \alpha_D V_{DS}}{1+C}
   + \frac{C}{1+C} (\frac{E_{F}}{q}-\frac{V_{DS}}{2})
 \end{equation}
 ここで
\begin{equation}
 C=\frac{q^2}{C_{eff}}\mathcal{D}_{2D} = \frac{q^2m^*}{C_{eff}\pi \hbar^2}
\end{equation}
である。
さらに実際top of the barrierにおける電子密度$N_{2D}$は
\begin{equation}
 \begin{split}
 N_{2D} &= \frac{1}{2}
  (n_{2D}(E_F-E_0) + n_{2D}(E_F-E_0-qV_{DS})) \\
  &\sim \frac{C C_{eff}}{q} (\frac{E_F - E_0}{q} - \frac{V_{DS}}{2}) \\
  &\propto \frac{1}{q}\frac{C}{1+C} \alpha_G C_{eff} V_{GS}
  = \frac{1}{q}\frac{1}{1+1/C} \alpha_G C_{eff} V_{GS}
 \end{split}
\end{equation}
これらの式から、有効質量$m^*$が大きくなった場合、
状態密度の増大によるキャリア密度の増大効果と、
Top of the barrierのエネルギー$E_0$の低下量が小さくなる効果がほぼ総裁し、
電流は増加する可能性があるものmerginal な増加であり、
実際には、injection velocityが低下すると考えられるため、
電流が減少すると予想される。

% $V_{GS}$ top of the barrier $E_0$の
% $V_{GS}$の増加に対する
% このため$V_{GS}$に対する電流の増加は小さくなることがわかる。

\appendix

 \begin{thebibliography}{99}
     
 \end{thebibliography}
%% \bibliography{/Users/motohisa/Dropbox/Papers/rse}

\end{document}

