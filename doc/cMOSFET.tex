\RequirePackage[l2tabu, orthodox]{nag}
\documentclass[11pt,uplatex,a4paper]{jsarticle}
%% file: fem2dp-2.tex
%% author: Junichi Motohisa
%% created: 2017-11-21 18:32:34
%% last saved : Time-stamp: <Fri Feb 28 18:28:27 JST 2020>
\usepackage[dvipdfmx]{graphicx}
\usepackage{multicol}
\usepackage{cite}
%\usepackage{mathptmx}
%\usepackage{color}
\usepackage{lscape}
% \usepackage{dcolumn}
\usepackage{moreverb}
% \usepackage[a4paper,margin=1in]{geometry}
\usepackage[driver=dvipdfmx,truedimen]{geometry}
% \geometry{left=30truemm,right=25truemm,top=25truemm,bottom=25truemm}
\geometry{margin=1truein}
% \usepackage[a4paper,margin=1in]{geometry}

%% taken from https://github.com/ryseto/emacs_and_tex/blob/master/template.tex
%% begin
\usepackage{amsmath,amsfonts,amsthm}
\usepackage{bm}
\usepackage[protrusion=true,expansion=true]{microtype}
\usepackage{siunitx}
\usepackage{xcolor}
%\usepackage[pdftex]{graphicx}
%\usepackage[pdftex,bookmarks,colorlinks]{hyperref}
% \usepackage[dvipdfmx]{hyperref}
% \usepackage{pxjahyper}
\usepackage[english]{babel}
\usepackage[T1]{fontenc}
\usepackage[utf8]{inputenc}
\usepackage{newtxtext}
\usepackage[varg]{newtxmath}
\usepackage{booktabs}
\usepackage{datetime}
\usepackage{setspace}
\usepackage{enumitem}
\usepackage{cleveref}
\usepackage[numbers]{natbib}

% \bibliographystyle{abbrvnat}
\setlist{itemsep=-3pt}
%% end

\def\d#1#2{\frac{d #1}{d #2}}
\def\dd#1#2{\frac{d^2 #1}{d #2^2}}
\def\pd#1#2{\frac{\partial #1}{\partial #2}}
\def\pdd#1#2{\frac{\partial^2 #1}{\partial #2^2}}
\def\Ene{\varepsilon}
\def\DOS{\mathcal{D}}
\def\vec#1{{\bm{#1}}}
\def\kvec{\bm{k}}
\def\kvecp{{\bm{k}'}}
\def\subvec#1{{\boldsymbol{\small #1}}}
 
\def\GP{{\vec{G}_\parallel}}
\def\GPP{{\vec{G}^\prime_\parallel}}
\def\KP{{\vec{k}_\parallel}}
\def\KP{{\vec{k}_\parallel}}
\def\XP{{\vec{x}_\parallel}}

\def\EneK{\varepsilon_{\kvec}}
\def\EneKp{\varepsilon_{\kvec^{'}}}
\def\EneKn{\varepsilon_{\kvec n}}
\def\EneKnp{\varepsilon_{\kvec^{'} n^{'}}}
\def\EneKm{\varepsilon_{m \vec{k}}}
\def\EneKmp{\varepsilon_{m' \vec{k}'}}

\def\bra#1{{\langle #1 |}}
\def\ket#1{{| #1 \rangle}}
\def\braket#1#2{{\langle #1 | #2 \rangle}}
\def\av#1{{\langle #1 \rangle}}

\def\sinc{\mathrm{sinc}}
\def\sgn{\mathrm{sgn}_{n,p}}
\def\Tr{{\mathrm{Tr}\, }}

\def\micon{\mathrm{\mu m}}
\def\nm{\mathrm{nm}}

\begin{document}

\title{Title}
\author{J.~Motohisa\\Graduate School of IST, Hokkaido University}
\date{\shortdate\today \, \ampmtime }
\maketitle

\section{Introduction}
\section{Models for Current-Voltage Characteristics of Nanowire FETs}
\label{sec:ivmodel}
\subsection{Explicit Continuous Model for Long-channel FETs}
In this appendix, we introduce two models for calculating current-voltage characteristics of NW-FETs.
The first one is explicit continuous model for long-channel undoped SG MOSFETs with
cylindrical cross section having radius $R$, which is derived by I\~n\'iguez \textit{et al.} \cite{Iniguez:2005ub}.
The drain current $I_{\mathrm{DS}}$ in their model is calculated by the following equation; 
%%%%%%%%%%%%% Eq. (9)
\begin{equation}
 I_{\mathrm{DS}} = \frac{2 \pi R}{L_{\mathrm{G}}} \mu
  \left[
   2 \varphi_{\mathrm{th}} (Q_{\mathrm{S}} - Q_{\mathrm{D}}) + \frac{Q_{\mathrm{S}}^2-Q_{\mathrm{D}}^2}{2 C_0}
   + \varphi_{\mathrm{th}} Q_0 \ln \left(\frac{Q_{\mathrm{D}} + Q_0}{Q_{\mathrm{S}}+Q_0}\right) 
  \right] \ , 
\end{equation}
where $L_{\mathrm{G}}$ is the gate length, $\mu$ is the mobility, and
\begin{equation}
Q_0=\frac{4 \varepsilon_{\mathrm{S}} \varepsilon_0}{R}\varphi_{\mathrm{th}} \ ,
\end{equation}
with $\varphi_{\mathrm{th}} = {k_B T}/{q}$,
and $\varepsilon_{\mathrm{S}}$ is the dielectric constant of the channel material.
$C_0$ is the oxide capacitance \textit{per unit area} (=$C_{OX}/2\pi R$ with $C_{OX}$ given by Eq. (\ref{eqn:Cox1})).
$Q_{\mathrm{D}}$ and $Q_{\mathrm{S}}$ are inversion charges at source and drain and determined by solving the following equation for $Q$
with $Q_{\mathrm{D}} = Q(V_{\mathrm{DS}})$ and $Q_{\mathrm{S}}=Q(0)$;
%%%%%%%%%%%% Eq. 6
\begin{equation}
 (V_{\mathrm{GS}} - \Delta \varphi -V) - \varphi_{\mathrm{th}}
  \ln \left(\frac{8}{\delta R^2}\right)
  = \frac{Q}{C_0} + \varphi_{\mathrm{th}} \ln \left(\frac{Q}{Q_0}\right) +
  \varphi_{\mathrm{th}} \ln \left(\frac{Q+Q_0}{Q_0}\right)
\end{equation}
Here, $\Delta \varphi$ is the difference of work function between channel material and the gate metal and
\begin{equation}
 \delta = \frac{q^2n_i}{k_B T \varepsilon_{\mathrm{S}}\varepsilon_0}
\end{equation}
with intrinsic carrier density $n_i$.
This equation should be solved numerically, but
a closed form for $Q$ is also given in Ref. \cite{Iniguez:2005ub} as
%%%%%%%%% Eq. (17)
\begin{equation}
 Q = C_0
  \left(
   -\frac{2 C_0 \varphi_{\mathrm{th}}^2}{Q_0} +
   \sqrt{\left(\frac{2 C_0 \varphi_{\mathrm{th}}^2}{Q_0}\right)^2
   + 4 \varphi_{\mathrm{th}}^2 \ln^2
   \left(
	1+ \exp \left(\frac{V_{\mathrm{GS}}-V_{\mathrm{T}}+\Delta V_{\mathrm{T}} -V}{2 \varphi_{\mathrm{th}}}
			\right)
   \right)
   }
  \right)
\end{equation}
with
% \begin{equation}
%  \varphi_{\mathrm{th}} = \frac{k_B T}{q}
  % \end{equation}
  %%%% Eq. 16
\begin{equation}
 V_{\mathrm{T}} = V_0 + 2 \varphi_{\mathrm{th}} \ln \left(1+\frac{Q'}{Q_0}\right)
\end{equation}
and
%%%% Eq. 18
\begin{equation}
 Q'=
  C_0
  \left(
   -\frac{2 C_0 \varphi_{\mathrm{th}}^2}{Q_0} +
   \sqrt{\left(\frac{2 C_0 \varphi_{\mathrm{th}}^2}{Q_0}\right)^2
   + 4 \varphi_{\mathrm{th}}^2 \ln^2
   \left(
	1+ \exp \left(\frac{V_{\mathrm{GS}}-V_0 -V}{2 \varphi_{\mathrm{th}}}
			\right)
   \right)
   }
  \right)
\end{equation}
and
\begin{equation}
 \Delta V_{\mathrm{T}} = \frac{2C_0\varphi_{\mathrm{th}}^2}{Q_0}
  \frac{Q'}{Q_0 + Q'}
\end{equation}
and
\begin{equation}
 V_0 = \Delta \varphi + \varphi_{\mathrm{th}} \ln \left(\frac{8}{\delta R^2}\right)
\end{equation}

\section{Velocity Saturation}
電流を与える\cite{Iniguez:2005ub}の式(7)は
\begin{equation}
 I_{DS} = \mu \frac{2 \pi R}{L}\int_0^{V_{DS}} Q(V) dV
\end{equation}
であるが、これは通常のMOSFETと同様に
\begin{equation}
 I_{DS} = 2 \pi R Q(V) \mu \d{V}{x}
\end{equation}
をチャネルに沿って$x$で積分したものと思われる。
したがって速度飽和を考慮するためには、例えばドリフト速度として
\begin{equation}
 v = \frac{\mu \d{V}{x}}{1+\frac{v_s}{\mu}\d{V}{x}}
\end{equation}
を考えてやると
\begin{equation}
 I_{DS} = 2 \pi R Q(V)  = 2 \pi R Q(V) \frac{\mu \d{V}{x}}{1+\frac{v_s}{\mu}\d{V}{x}}
\end{equation}
となる。分母をはらって$x$で積分すると
\begin{equation}
 I_{DS} (L +  \frac{v_s}{\mu} \int_0^{L} dx \d{V}{x})
  = \mu 2 \pi R \int_0^{L} dx Q(V) \d{V}{x}
\end{equation}
となるので
\begin{equation}
 I_{DS} = \mu \frac{2 \pi R}{L} \frac{1}{{1 + \frac{v_s V_{DS}}{\mu L}}}
  \int_{Q_S}^{Q_D} Q(V) dV
\end{equation}
となる。

\appendix

 \begin{thebibliography}{99}
  \bibitem{Iniguez:2005ub}
		  Iniguez, B., Jimenez, D., Roig, J., Hamid, H. A., Marsal, L. F., and Pallares, J. (2005). Explicit continuous model for long-channel undoped surrounding gate MOSFETs. IEEE Transactions on Electron Devices, 52(8), 1868–1873.     
 \end{thebibliography}
%% \bibliography{/Users/motohisa/Dropbox/Papers/rse}

\end{document}

